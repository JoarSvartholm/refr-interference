\section{Theory}

\begin{equation}
  \label{eq:OPD}
  \tn{OPD} = 2\tn{OPL}_{L_2} - 2\tn{OPL}_{L_1} = 2nd + \tn{const.}
\end{equation}

where OPL is the optical path length

\begin{equation}
  \label{eq:interference}
  \tn{OPD} = N \lambda
\end{equation}

where $N$ is the number of interference fringes and $\lambda$ is the wavelength of the light. Putting this into Eq. \eqref{eq:OPD} yields

\begin{equation*}
  N \lambda = 2nd + \tn{const.}.
\end{equation*}

Instead of counting the absolute number of fringes one can count the number of fringes that appear when changing the refractive index of the sample, namely by increasing the pressure from vacuum then

\begin{equation*}
  \Delta N \lambda = 2d \Delta n = 2 d \alpha \frac{\Delta P}{P_0},
\end{equation*}

where $P$ is the absolute pressure,$P_0$ is the pressure according to STP conditions and $\alpha$ 


