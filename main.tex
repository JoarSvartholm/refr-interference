\documentclass[a4paper,11pt]{article}
\usepackage[utf8]{inputenc}            % Tekenkodning
\usepackage[T1]{fontenc}               % Fixa kopiering av texten
\usepackage[english]{babel}            % Språk (t.ex. Innehåll)
\usepackage{geometry}                  % Sidlayout m.m.
\usepackage{graphicx,epstopdf,float}   % Bilder
\usepackage{amsmath,amssymb,amsfonts}  % Matematik
\usepackage{enumerate}                 % Fler typer av listor
\usepackage{fancyhdr}                  % Sidhuvud/sidfot
\usepackage{hyperref}                  % Hyperlänkar
\usepackage{parskip}                   % noindent!!
\usepackage{float}                     % \begin{figure}[H] preciserar bildposition
\usepackage{subcaption}             % För att lägga figurer bredvidvarandra från: http://tex.stackexchange.com/questions/91224/placing-two-figures-side-by-side
\usepackage{textcomp}
\usepackage{gensymb}
\usepackage{wrapfig}

% Add new commands used-defined:
\newcommand{\m}[1]{\mathbf{#1}}
\newcommand{\tn}[1]{\textnormal{#1}}


% instälningar för figurtexter
\usepackage[margin=3ex,font=it,labelfont=bf,labelsep=endash]{caption}

% mailadresser som hyperlänkar
\newcommand{\mail}[1]{\href{mailto:#1}{\nolinkurl{#1}}}
% Spara författare och titel
\let\oldAuthor\author
\renewcommand{\author}[1]{\newcommand{\myAuthor}{#1}\oldAuthor{#1}}
\let\oldTitle\title
\renewcommand{\title}[1]{\newcommand{\myTitle}{#1}\oldTitle{#1}}

% Hyperlänkar
\hypersetup{
  colorlinks   = true, %Colours links instead of ugly boxes
  urlcolor     = black, %Colour for external hyperlinks
  linkcolor    = black, %Colour of internal links
  citecolor   = black %Colour of citations
}



\graphicspath{{./images/},{./matlab}} % Söker också bilder i en undermapp figs.


%% DOCUMENT
%------------------------------------------------------------------%
\begin{document}
  \title{Crack Detection using Inductive Methods}


  \author{
    Måns Bermell (\mail{mabe0395@student.umu.se})\\
    Joar Svartholm (\mail{josv0150@student.umu.se})
    }


  \date{\today}


\begin{titlepage}
  \maketitle
  \thispagestyle{fancy}
  \headheight 35pt
  \rhead{\small\today}
  \lhead{\small Department of Physics\\
    Umeå University}


% State the aim of the experiment, what was measured, which techniques and methods were used, and the main result(s) and conclusion(s). Remember that the abstract should be understandable on its own, and you can thereby not refer to equations/figures/tables in the report. You should also not use references, since the information in the abstract should be available in the actual report.
\vspace{1cm}
\begin{abstract}
\noindent
The Nobel prize in physics went in 2017 to the group who confirmed the prescence of gravitational waves. This was discovered using a type of a Michelson interferometer sensing small changes in space. In this report a Michelson interferometer was built in order to measure the refractive index of different gases. This could be done since light travels at different velocities in different materials. A change in refractive index, i.e. a change in pressure of a gas could be measured using this device with high accuracy. The refractive index of air was determined to 1.000264(6) which can be compared to the tabulated value of 1.000271. Similarly for argon and nitrogen the refractive indices was slightly below the tabulated values. This could be due to a systematic error in the calibration or an underestimation of the error. Therefractive index of helium was calculated to 1.000034(4) which is higher that the tabulated value. This could be due to that the relative error in this is higher than the others which could make the systematic error invisible.


\end{abstract}

  % Ändra till rätt namn m.m.
  \cfoot{Non-invasive measurement techniques / v1.0\\
    Supervisor: Piotr Matyba }

\end{titlepage}


\newpage
\pagestyle{fancy}
\headheight 30pt
\rhead{\small \myTitle\\\today}
\lhead{\small \myAuthor}
\cfoot{\thepage}

% Innehåll
%\tableofcontents

%\newpage
\section{Introduction}

%TODO: Vad tycker du???

The 2017's Nobel price in physics was awarded to the group who confirmed Einstein's theory about gravitational waves. These waves was detected via an enormous interfermoeter. An interfermoeter can measure very small changes in optical paths and works by utilizing coherant light and the physical phenomena of interference. 

Light is the fastest moving thing know to humans, it travels at $299.792.458$ m/s in vaccum. This speed actually depends on the matter i travels in, for example, in water the speed is only $75\%$ compared to vaccum. These different speeds are inversly proportional to a materials index of refraction. 

In this report, the index of refraction is calculated for air, helium, argon and nitrogen. These were determined by examining the gases at different pressures in an Michelson interferometer.


\section{Theory}

Light propagates through a medium at a lower velocity than in vaccum. This is described by the index of refraction of the medium

\begin{equation}
  \label{eq:refrInd}
  n = \frac{c_0}{v},
\end{equation}

where $c_0$ is the speed of light in vaccum and $v$ is the velocity in the medium\cite{phH}. By this one can define the optical path length (OPL) as the length the light experiences, that is if the light propagates through a medium of length $L$ and refractive index $n$, then the OPD will be

\begin{equation*}
  \tn{OPD} = nL.
\end{equation*}

The refractive index is thus a measure of how dense a material is and must therefore be proportional to the pressure. Namely by increasing the pressure $P$ from vaccum, then

\begin{equation}
  \label{eq:refrVaccum}
  n-1 = \alpha \frac{\Delta P}{P_{atm}},
\end{equation}

where $P_{atm}$ is the atmospheric pressure and $\alpha$ is the proportionality constant. Note thet since $P$ is measured from vaccum then $\alpha=n-1$ at atmospheric pressure.

\subsection{The Michelson Interferometer}
A Michelson interferometer consists of a laser beam being splitted into a reference leg $L_1$ and a signal leg $L_2$. The beams are then reflected back and reassembled at a sensor. A schematic sketch of this is found in Fig. \ref{fig:experimentalSetup}. If the beams are aligned correctly they will interfere with eachocher and constructive interference will occur if

\begin{equation}
\label{eq:interference}
  \tn{OPD} = N \lambda,
\end{equation}

where OPD is the optical path difference between the legs, $N$ is an integer and $\lambda$ is the wavelength of the light.

\subsection{Method}

If a gas chamber of length $d$ is put on the sensor leg as is shown in Fig. \ref{fig:experimentalSetup} the the OPD can be found as

\begin{equation}
  \label{eq:OPD}
  \tn{OPD} = 2\tn{OPL}_{L_2} - 2\tn{OPL}_{L_1} = 2nd + \tn{const.}
\end{equation}

Using the interference condition Eq. \eqref{eq:interference} this can be reduced to

\begin{equation*}
  N \lambda = 2nd + \tn{const.}.
\end{equation*}

Instead of counting the absolute number of interference maxima, called fringes, one can count the number of fringes that appear when changing the refractive index of the sample. This can then be rewritten as

\begin{equation}
\label{eq:delta}
  \Delta N \lambda = 2d \Delta n.
\end{equation}

If one measures this from vaccum one can make us of Eq. \eqref{eq:refrVaccum} in Eq. \eqref{eq:delta} and obtain

\begin{equation}
\label{eq:slope}
  \Delta N = \frac{2d\alpha}{\lambda P_{atm}} \Delta P.
\end{equation}

Thus, by increasing the pressure of a gas from vaccum and counting how many fringes that has passed a sensor one can find the refractive index from the slope of Eq. \eqref{eq:slope} and $n=\alpha+1$.


\section{Experimental setup}
% TODO: fixa röda tråden mellan metoden och utförandet.

The experimental setup used to measure the index of refraction for various gases is illustrated in Fig. \ref{fig:experimentalSetup}. The HeNe-laser beam is divided into two legs, reference leg $L_1$ and signal leg $L_2$ via a beam splitter (BS). The ray travelling down the signal leg is transmitted through a gas chamber made of acryllic glas, with inner length $d=100(1)$ mm. It is then reflected via a mirror back through the gas chamber and reflected on the beam splitter where it coincides with the other part of the ray (from the reference leg). The recombined laser beam propagates through a small apperature in order to elliminate unwanted reflections created by the equipment. The united ray also reaches a lens which diverge the rays in order to increase the resolution of the interference pattern obtained. A photodiode is mounted after the lens in order to pick up any changes of incident intensity of the combined laser beam. In order to eliminate ethalons inside the gas chamber, it was mounted at an angle relative the incident laser beam.

\begin{figure}[H]
  \centering
  \includegraphics[width=0.8\textwidth]{Exp_setup.png}
  \caption{Experimental setup. The laser beam is split into two rays at the beam splitter. One ray propagates in the reference leg (L$_1$) and is reflected back to the beam splitter. The other ray that was split at the beam splitter, propagates in to the signal leg (L$_2$) via a gas chamber to a mirror and reflected back the same way to the beam splitter. The two rays coincide at the beam splitter again and propagates through an apperature and a lens and illuminates a photodiode. The two rays will interfere with eachother and an interferance pattern will emerge.}
  \label{fig:experimentalSetup}
\end{figure}

The photodiode is connected to an operational amplifier with a low pass filter in order to amplify the signal and reduce noise in the circut. The data acquisition card (DAQ) retrives the amplified signal from the photodiode as well as a signal from an instrument measuring the pressure inside the gas chamber. The data is captured via LabView and analyzed in MATLAB.

In order to convert the voltage accuired from the pressure measurment into pressure, it had to be calibrated. The calibration curve is shown in Fig. \ref{fig:calibration}. The linear fit on the calibration measurement was calculated in by linear least squares in MATLAB, the fitted line has the equation $P = 12.5(2)V+25.6(1)$.


\begin{figure}[H]
  \centering
  \includegraphics[width=0.8\textwidth]{matlab/calibration.png}
  \caption{Calibration curve for voltage to pressure relation. The fitted line is $P = 12.5(2)V+25.6(1)$.}
  \label{fig:calibration}
\end{figure}

The two legs were aligned propertly in order to see a interference pattern at the photodiode with high resolution and contrast. Filling the gas chamber with one of the gases of intrest (air, helium, argon or nitrogen), from vaccum and measuring the change of intensity at the photodiode as well as the pressure in the gas chamber continuously. Then by counting the fringes (seen as peaks from the photodiode). The index of refraction could be calculated via Eq. \ref{eq:refrVaccum} and Eq. \ref{eq:slope}.

%, an interferance pattern will emerge on the photodiode. If vaccum is created inside the gas chamber and then a steady flow of gas added, the optical path will increase. An interferance pattern (fringes) on the photodiode  will start to move continuously. By measuring the intensity changes on the photodiode and relating this to the pressure change in the gas chamber, one can calculate the index of refraction via Eq. \ref{eq:refrVaccum} and Eq. \ref{eq:slope}.



In order to estimate the error of $a =\Delta N/ \Delta P$ from Eq. \ref{eq:slope}, the errors of $\Delta N$ ($S_{\Delta N}$) and $\Delta P$ ($S_{\Delta P}$) had to be estimated. These errors was estimated to $S_{\Delta N}=0.5$ and $S_{\Delta P}=0.1$ kPa. By fitting two lines that are inside these errors for all measurements, with as large slope and small slope as possible. These slopes extream slopes will be used as the upper and lower limits on $a$.


\section{Results and discussion}
The number of fringes was plotted against the pressure change for the different gases and is found in Fig. \ref{fig:measurements}. As one can see in the figures all gases showed a linear relation which was expected. When comparing Fig. \ref{fig:Helium} with the others the number of fringes is a lot lower than for the others. This means that the index of refraction must be a lot lower for this and the relative error will probably be larger. It also means that the relative error in the result will be quite larger in this result in comparison to the others.

The linear fit parameters are found in Tab. \ref{tab:linearFits}. As expected, the slope $a$ for helium is a lot lower than the other slopes. From these slopes, the refractive indexes was estimated using Eq. \eqref{eq:slope} and tabulated in Tab. \ref{tab:refrIndex}.

\begin{figure}[H]
  \centering
  \begin{subfigure}{0.49\textwidth}
    \includegraphics[width=\textwidth]{matlab/Air}
    \caption{Air}
    \label{fig:Air}
  \end{subfigure}
  \begin{subfigure}{0.49\textwidth}
    \includegraphics[width=\textwidth]{matlab/Helium}
    \caption{Helium}
    \label{fig:Helium}
  \end{subfigure}
  \begin{subfigure}{0.49\textwidth}
    \includegraphics[width=\textwidth]{matlab/Argon}
    \caption{Argon}
    \label{fig:Argon}
  \end{subfigure}
  \begin{subfigure}{0.49\textwidth}
    \includegraphics[width=\textwidth]{matlab/Nitrogen}
    \caption{Nitrogen}
    \label{fig:Nitrogen}
  \end{subfigure}
  \caption{Number of fringes and pressure change for the different gases with linear fits. Fitting parameters are found in Tab. \ref{tab:linearFits}}
  \label{fig:measurements}
\end{figure}

\begin{table}[H]
  \centering
  \caption{Fitting parameters for the measured values in Fig. \ref{fig:measurements}. Linear fit on the form $y=\alpha x + \beta$.}
  \label{tab:linearFits}
  \begin{tabular}{l|l|l}
        & $\alpha$ & $\beta$ \\ \hline
  Air   & $0.828(1)$ kPa$^{-1}$ & 0.16(6) \\
  Helium & $0.108(1)$ kPa$^{-1}$ & 0.06(5) \\
  Argon  & $0.793(1)$ kPa$^{-1}$ & 0.61(5) \\
  Nitrogen & $0.843(1)$ kPa$^{-1}$ & 0.86(5)
  \end{tabular}
\end{table}

\begin{table}
  \centering
  \caption{Estimated refractive indexes for the gases including tabulated values}
  \label{tab:refrIndex}
  \begin{tabular}{l|l|l|l}
          & measured & tabulated & error \\ \hline
    Air     & 1.00026410 & 1.00027116 & 7.05654305e-06 \\
    Helium  & 1.00003440 & 1.00003233 & 2.07707264e-06 \\
    Argon   & 1.00025283 & 1.00026106 & 8.23800429e-06 \\
    Nitrogen& 1.00026891 & 1.00027646 & 7.55338923e-06
  \end{tabular}
\end{table}


\section{Conclusion}


The measured refractive indices was slightly off the tabulated. All but the one for helium was lower than the tabulated value which indicates that there might have been a systematic error in the measurents. This might be due to that the vaccum pump wasn't able to create perfect vaccum and thus introduced a shift. This was not seen in the measurements of helium since the relative error of the other error sources was larger and thus making this shift invisible in the results.



\begin{thebibliography}{}

\bibitem{phH} Nordling, C., 2006. Physics Handbook for Science and Engineering.
\end{thebibliography}
%
% \include{appendix}



\end{document}
