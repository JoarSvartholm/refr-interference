The Nobel prize in physics went in 2017 to the group who confirmed the prescence of gravitational waves. This was discovered using a type of Michelson interferometer sensing small changes in space. In this experiment, a Michelson interferometer was built in order to measure the refractive index of different gases. This could be done since light travels at different velocities in different materials. A change in refractive index, i.e. a change in pressure of a gas could be measured using this device with high accuracy. The refractive index of air was determined to 1.000264(6) which can be compared to the tabulated value of 1.000271. Similarly for argon and nitrogen the refractive indices were slightly below the tabulated values. This could be due to a systematic error in the calibration or an underestimation of the error. The refractive index of helium was calculated to 1.000034(4) which is higher that the tabulated value of 1.000032.
